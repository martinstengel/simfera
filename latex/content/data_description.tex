

\subsection{ERA-Interim reanalysis}\label{sec:era}

The ERA-Interim global atmospheric reanalysis \cite{Dee2011} provided by ECMWF 
(European Centre for Medium-Range Weather Forecasts) 
is the follow-up of ERA-40 reanalysis \cite{Uppala2005}, which was completed in 2002.
It covers the period from 1979 onwards and is continuously extended in near-real time.
One of the main objectives was to solve various difficulties regarding data assimiliation
(e.g. use of satellite data), which were found during the production of ERA-40. 
Evaluations of the new dataset have demonstrated that the hydrological cycle, the stratospheric circulation, and
the consistency in time of the reanalysis fields have been improved.
Data assimilation methodology, forecast model, and input observations
are essential components for the generation of high quality reanalysis data,
and consequently, will lead to more confident results in climate change applications. 

The reanalysis is produced by the Integrated Forecast System (IFS), which includes
the forecast model consisting of three fully coupled components for the atmosphere, land surface, and ocean waves.
ERA-Interim clouds are represented by a fully prognostic cloud scheme where cloud-related processes
are treated in a unified way, i.e. they are physically realistic and consistent with the rest of the model.
The scheme for stratiform and convective clouds has been developed by Tiedtke \cite{Tiedtke1993}, 
implemented in the global forecast model operationally in 1995 and since then, continuously improved.
An important and indispensable limitation of large-scale models is the fact that only bulk
properties of clouds can be taken into account. Hence, clouds are defined by
the horizontal coverage of the grid box by cloud, and the mass mixing ratio of total cloud condensate,
along with the constraint that cloud air is saturated with regard to water and ice, respectively.
The thermodynamic phase is determined on the basis of the temperature.
In other words, cloud cover and cloud water/ice content are derived from prognostic equations,
which follow the mass balance equation for cloud water/ice content and cloud air.
As time evolves in the simulation the cloud variables are changing due to source and
sink terms that are related to cloud formation (e.g., condensation/sublimation, cumulus convection)
and destruction (e.g., evaporation, precipitation) processes, respectively.

% % % A new prognostic bulk microphysics scheme for the ECMWF forecast model
% % % R. M. Forbes (1), A. M. Tompkins (2), and A. Untch (1)
% % % (1) ECMWF, shinfield Park, Reading U.K, (2) ICTP, Earth System Physics, Trieste, Italy (tompkins@ictp.it)
% A major upgrade to the parametrization of stratiform cloud and precipitation was implemented in the Integrated
% Forecast System (IFS) cycle 36r4, operational at ECMWF from 9 November 2010. Three additional prognostic
% variables  have  been  introduced  to  enable  a  more  physically  based  representation  of  mixed-phase  (liquid/ice)
% cloud and precipitating rain and snow. A fully implicit method is employed to solve the network of microphysics
% pathways stably for long timesteps. It is the most significant change to the structure of the cloud parametrization
% since the Tiedtke scheme was introduced operationally in 1995. Many aspects of the model are systematically
% improved including the skill of precipitation forecasts, the spatial distribution of ice and snow in the troposphere,
% the physical processes in mixed-phase cloud and the impact of cloud and precipitation on radiation.

As already metioned above the original Tiedtke scheme has been continuously further developed
aiming at a more physically realistic representation of cloud and precipitation microphysics
that cannot be achieved by using just two prognostic cloud variables.
A revised Tiedtke cloud parameterization of stratiform cloud and precipitation became available by 2010
increasing the number of prognostic variables from two (cloud fraction, cloud condensate) to five
(cloud fraction, cloud liquid water, cloud ice, rain, and snow).
The updated approach treats now water and ice clouds independent leading to more realistic
modelling of supercooled liquid water clouds as well as snow and rain precipitation processes.

% % % taken from Dee et al. 2011 % % %
% Improvements of the model physics have caused a positive impact on the 
% representation of clouds in the ERA-Interim dataset \cite{Dee2011}. By comparing the ERA-Interim and ERA-40 
% reanalysis with ISCCP observations several improvements regarding cloud cover have been identified. 
% For instance, the marine stratocumulus cloud cover has been increased by 15 -- 25~\%
% because a new moist boundary-layer scheme was implemented \cite{Koehler2011}.
% The tropical ocean total cloud cover has been decreased by 5 -- 15~\% 
% due to an overall improved hydrological cycle and the incorporation of ice supersaturation 
% delaying the formation of ice clouds \cite{Tompkins2007}.
% The cloud cover over land in the tropics has been increased by 20 -- 30~\% caused by an
% increase in high cloud resulting from improved deep convective triggering \cite{Bechtold2004} along
% with low cloud produced by a new boundary-layer scheme.
% The midlatitude ocean (low- and medium-height) cloud cover has been increased by about 5~\% due to
% improved numerics of the cloud scheme.

Models and observations are incomplete having uncertainties. Data assimiliation is the process,
which incorporates observations into the model regime generating the optimum atmospheric 
state at a particular time, which is better than that from model or observation on their own.
As a consequence, random errors can be minimized. However, there are also many systematic 
differences between models and observations, which require a careful treatment \cite{Forbes2015}.

\cite{IPCC2013}
In the last two decades significant progress have been made in improving the parameterization 
of clouds in large-scale models. But still assumptions and simplifications are necessary 
limiting the
% In any large-scale model basic assumptions and simplifications are necessary for representing clouds.
% Clouds and also aerosol are still the largest uncertainties in GCMs and thus,
% it is important to further understand the processes and improve their representation in models.
% altough in den last 10 years major progress has been made
% models needs to be evaluated by other models and observations
% retrievals needs to be verified by models
% neither model nor retrievals are perfect due to the complexity of the real world.
% assumptions and limitations are required in both
% make bridge to CC4CL here


\subsection{Cloud\_cci dataset}\label{sec:cloudcci}

The ESA Cloud\_cci project presents new long-term coherent AVHRR and MODIS cloud property datasets 
(1982 – 2014) along with associated uncertainties obtained by optimal estimation theory. 
The algorithm fits a physically consistent cloud/atmosphere/surface model to satellite observations 
simultaneously from the visible to the mid-infrared, thereby ensuring spectral consistency for 
all retrieved and derived parameters. This retrieval scheme, referred to as Community Optimal 
Estimation Cloud Retrieval for Climate (CC4CL), provides cloud fraction, cloud top level estimates, 
cloud thermodynamic phase, cloud optical thickness, cloud effective radius and post processed products, 
such as cloud liquid and ice water path.
