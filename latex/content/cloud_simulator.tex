

This section describes the main steps of retrieving monthly mean pseudo-satellite 
observations derived from global atmospheric reanalysis produced by ECMWF.
The clouds simulator reads the ERA-Interim 6-hourly (00, 06, 12, 18 UTC) 
gridded estimates of three-dimensional (3D) meteorological upper upper air parameters on 60 model levels
with the top of the atmosphere located at 0.1 hPa:
% ERA-Interim files per month containing the following upper air parameters on model levels:
\begin{itemize}
    \setlength\itemsep{0.2em}
    \item liquid water content ``LWC'' 
          [kg/kg]$\footnote{mass of condensate per mass of moist air}$,
    \item ice water content ``IWC'' [kg/kg],
    \item cloud cover ``CC'' (0-1),
    \item temperature ``T'' [K], and
    \item specific humidity ``Q'' [kg/kg].
\end{itemize}
Additionally, the files comprise the surface geopotential ``Z'' [m$^{2}/s^{2}$] and 
the logarithm of surface pressure ``LNSP'' [Pa] for each grid box. 
These are not strictly upper air parameters, however, they are needed to compute
the vertical pressure and geopotential profiles by using the provided ``A'' and ``B'' 
coefficients on model levels along with the temperature and specific humidity profiles.

The following steps address the things needed to simulate the observational process.
In other words, what would a satellite (e.g., MODIS, AVHRR) 
see if the atmosphere had the clouds of a climate model? 
% What would CC4CL retrieve in such cases?


% ==================================================================================
\subsection{Pre-processing}\label{sec:preproc}
% ==================================================================================

In the first step the grid box mean liquid and ice water content at each model level
are weighted by the cloud cover profile providing the so-called 
in-cloud liquid and ice water content. 
Thus, only the cloudy part of the grid cell is taken into account, 
which is comparable to what a satellite would detect.
Based on these cloud-cover-weighted profiles the liquid and ice water paths
per model layer can be derived, which present the total amount of liquid and ice
water between two consecutive model levels in the atmosphere, respectively.
The total amount of liquid and ice water path is referred to as cloud water path (CWP).

The cloud optical thickness (COT) per layer is obtained by the method of \citet{Han1994}
\begin{equation}\label{eq:han}
    COT = \frac{3}{4} \frac{CWP \cdot  Q_{ext}}{CER \cdot \rho}
\end{equation}
where CWP represents either the liquid or ice water path depending on the thermodynamic phase.
$Q_{ext}$ denotes the extinction coefficient, which is assumed to be 2 for water and 2.1 for ice.
The density $\rho$ is set to 1 $g/m^{3}$ for water and 0.9167 $g/m^{3}$ for ice.
For the computation of effective radii (CER) per layer the 
ERA-Interim parameterizations are implemented in the cloud simulator described below.

The cloud droplet effective radius follows the method of \citet{Martin1994} and 
is defined as function of liquid water content and cloud droplet number concentration, 
which in turn depends on the wind speed. For practical reasons the simulator uses constant 
values for the number of cloud condensation nuclei over land (300) and sea (100).
Hence, a land-sea mask (Fig.~\ref{fig:lsm}) is required, 
which is obtained from ERA-Interim Sea Surface Temperature (Fig.~\ref{fig:sst}) 
data aggregated on a 0.5$^{\circ}$ latitude-longitude grid.
The ice crystal effective radius is parameterized as a function of 
temperature and ice water content based on \citet{Sun1999}, 
which has been revised by \citet{Sun2001}.

\begin{figure}[!htp]
  \begin{minipage}[t]{0.5\textwidth}
    \subfloat[ERA-Interim land-sea mask (LSM).]
    {\includegraphics[scale=0.26]{./figures/{lsm_era_interim_0.5_0.5}.png}\label{fig:lsm}}
  \end{minipage}
  \begin{minipage}[t]{0.5\textwidth}
    \subfloat[ERA-Interim sea surface temperature (SST).]
    {\includegraphics[scale=0.26]{./figures/{sst_era_interim_0.5_0.5}.png}\label{fig:sst}}
  \end{minipage}
  \caption[ERA-Interim land-sea mask and sea surface temperature (31/08/2015).]
{ERA-Interim auxiliary data provided on a 0.5$^{\circ}$ latitude-longitude grid.}
\label{fig:lsm_sst}
\end{figure}

In CC4CL the simultaneously retrieved cloud optical thickness and 
cloud effective radius are combined utilizing the relationship given in 
Eq.~\ref{eq:han} for deriving cloud water path as post-processed product.
Thus, the cloud simulator adopts this retrieval characteristic of CC4CL by
using this equation for computing cloud optical thickness per model layer 
since ERA-Interim provides solely the information on the cloud water path and cloud cover.




% ==================================================================================
% % % INPUT  = grid box mean profiles (nlev)
% % % OUTPUT = grid box matrices (ncol, nlev)
\subsection{Downscaler}\label{sec:downscaler}
% ==================================================================================

In the next step the simulator adresses the mismatch in scale between 
that of a ERA-Interim model grid box ($\sim$ 500 km) and 
that of a satellite footprint ($\sim$ 5 km, e.g. AVHRR).
This is achieved by scaling vertical profiles for each model grid cell down to 
subcolumns$\footnote{
Currently each grid cell is broken into 20 subcolumns. This will be changed 
so that the number of subcolumns will be a function of latitude.}$, which
can be thought of as respresenting the spatial resolution of the instrument.
The downscaler assumes maximum-random overlap of clouds \citep{Geleyn1979}, which means 
that clouds in neighboring layers are maximally overlapped, 
while groups of clouds separated by one or more clear layers are randomly overlapped.

% grid box profiles 
\begin{figure}[!ht]
  \centering
  \begin{minipage}{0.5\textwidth}
    \subfloat[CFC profile.]
    {\includegraphics[scale=0.25]{{\snapcase_cfc_profile}.png}\label{fig:cfc_profile}}
  \end{minipage}\hfill
  \begin{minipage}{0.5\textwidth}
    \subfloat[COT profile.]
    {\includegraphics[scale=0.25]{{\snapcase_cot_profile}.png}\label{fig:cot_profile}}
  \end{minipage}\hfill
  \begin{minipage}{0.5\textwidth}
    \subfloat[CER profile.]
    {\includegraphics[scale=0.25]{{\snapcase_cer_profile}.png}\label{fig:cer_profile}}
  \end{minipage}\hfill
    \begin{minipage}{0.5\textwidth}
    \subfloat[CWP profile.]
    {\includegraphics[scale=0.25]{{\snapcase_cwp_profile}.png}\label{fig:cwp_profile}}
  \end{minipage}
  \caption[Vertical profiles on model levels.]{Grid box mean profiles of 
  cloud fraction (a), cloud optical thickness (b), cloud effective radius (c), 
  and cloud water path (d). The grid cell is located at {\longitude} and {\latitude} 
  on {\gcdate} at {\gchour}.}\label{fig:profiles}
\end{figure}

% grid box matrices
\begin{figure}[!ht]
  \centering
  \begin{minipage}{0.5\textwidth}
    \subfloat[CFC matrix.]
    {\includegraphics[scale=0.26]{{\snapcase_cfc_matrix}.png}\label{fig:cfc_matrix}}
  \end{minipage}\hfill
  \begin{minipage}{0.5\textwidth}
    \subfloat[COT matrix.]
    {\includegraphics[scale=0.26]{{\snapcase_cot_matrix}.png}\label{fig:cot_matrix}}
  \end{minipage}\hfill
    \begin{minipage}{0.5\textwidth}
    \subfloat[CER matrix.]
    {\includegraphics[scale=0.26]{{\snapcase_cer_matrix}.png}\label{fig:cer_matrix}}
  \end{minipage}\hfill
  \begin{minipage}{0.5\textwidth}
    \subfloat[CWP matrix.]
    {\includegraphics[scale=0.26]{{\snapcase_cwp_matrix}.png}\label{fig:cwp_matrix}}
  \end{minipage}\hfill
  \begin{minipage}{0.5\textwidth}
    \subfloat[CPH matrix.]
    {\includegraphics[scale=0.26]{{\snapcase_cph_matrix}.png}\label{fig:cph_matrix}}
  \end{minipage}\hfill
  \begin{minipage}{0.45\textwidth}
  \caption[Down-scaled cloud parameter profiles.]{Subcolumn distribution 
  of cloud fraction (a), optical thickness (b), effective radius (c), water path (d), and 
  phase (e) derived from grid box mean profiles shown in Figure~\ref{fig:profiles}. 
  The cloud phase represents the liquid cloud fraction, 
  i.e. the ratio of liquid water path and total cloud water path.
  The grid cell is located at {\longitude} and {\latitude} on {\gcdate} at {\gchour}.}
  \label{fig:matrices}
  \end{minipage}
\end{figure}

Figure~\ref{fig:profiles} and Figure~\ref{fig:matrices} 
represent the input and output, respectively, of the downscaler
for a grid cell located at {\longitude} and {\latitude} (i.e. near {\gccity}, {\gcland})
on {\gcdate} at {\gchour}. UTC is the official abbreviation for ``Coordinated Universal Time''.
A pseudo-sampling process takes the grid box mean vertical profile of the cloud fraction as input 
(Fig.~\ref{fig:cfc_profile}) and generates a cloud fraction matrix 
(Fig.~\ref{fig:cfc_matrix}) arranged in 20 subcolumns and 60 model levels.
Each element of the matrix is either cloudy (1) or cloud free (0).
For instance, at around 500~hPa the cloud cover is about 75\% and therefore,
15 subcolumns at this model level are assigned to 1.
Then the grid box mean profiles of cloud optical thickness (Fig.~\ref{fig:cot_profile}),
cloud effective radius (Fig.~\ref{fig:cer_profile}), 
and cloud water path (Fig.~\ref{fig:cwp_profile}) are used
to generate the corresponding matrices by filling the cloudy elements
in each layer with the respective value of the grid box profile 
(see Fig.~\ref{fig:cot_matrix}, \ref{fig:cer_matrix}, and \ref{fig:cwp_matrix}).
The cloud phase matrix (Fig.~\ref{fig:cph_matrix}) is calculated as the ratio
of liquid water path to cloud water path, and thus represents 
the fraction of liquid water clouds (0=ice, 1=liquid).
% 1 liquid, 0 ice, in between mixed

The used approach for deriving subgrid profiles is very similar to
SCOPS (Subgrid Cloud Overlap Profile Sampler, \citet{Webb2001}),
which is implemented, for instance, in COSP (CFMIP Observation Simulator
Package, \cite{Bodas2011}).
COSP is an integrated satellite simulator, which has been developed by 
the Cloud Feedback Model Intercomparison Project (CFMIP) community.
It is capable to simulate observations of multiple active and passive
satellite instruments (e.g., CloudSat, Calipso, ISCCP, MISR, MODIS, RTTOV) 
and therefore, allows the quantitative evaluation of clouds, humidity,
and precipitation processes in diverse numerical models.
For testing and verifcation purposes SCOPS has been implemented in the cloud simulator. 
The comparison of two runs, one based on the downscaler and the other one using SCOPS,
led to similar monthly mean results.





% ==================================================================================
% % % INPUT  = grid box matrices (ncol, nlev)
% % % total COT used for detecting the cloud from top to bottom using a thv value of e.g. 0.15
% % % OUTPUT = grid box arrays (ncol)
\subsection{Pseudo-retrieval}\label{sec:pseudo-retrieval}
% ==================================================================================

The grid box matrices (i.e. subcolumn cloud parameter profiles obtained from 
the downscaler, see Fig.~\ref{fig:matrices}) are then passed to the pseudo-retrieval module.
In each subcolumn it searches top down for the uppermost cloud layer where the 
total (i.e. vertically integrated) cloud optical thickness 
exceeds a certain threshold for the first time. 
From this model level all subcolumn parameters are collected 
either from grid box mean profiles or matrices, as explained below by means of an example.


% cloud top parameters 
\begin{figure}[!ht]
  \begin{minipage}{\textwidth}
    {\includegraphics[width=\textwidth]{{\snapcase_retrieved}.png}}
  \end{minipage}
  \caption[Retrieved subcolumn cloud top parameters.]
{Retrieved subcolumn cloud fraction (CFC), cloud top pressure (CTP),
cloud top temperature (CTT), cloud top height (CTH),
cloud phase (CPH), cloud effective radius (CER),
cloud optical thickness (COT), and cloud water path (CWP) 
for one grid box. The grid cell is located at
{\longitude} and {\latitude} on {\gcdate} at {\gchour}.
The grid box average is given on top of each panel.}
\label{fig:retrieved}
\end{figure}

Figure~\ref{fig:retrieved} shows the derived cloud top parameters for each subcolumn, 
where the total cloud optical thickness is larger than 0.15. 
In the first upper left panel one can see that for each subcolumn a binary cloud mask 
is returned, i.e. the cloud fraction is set to 1 (cloudy) or to 0 (cloud-free).
The cloud top pressure, temperature, and height (remaining upper panels from left to right) 
are derived from the pressure, temperature, and geopotential model profiles.
The lower panels from left to right show the cloud phase, effective radius, optical thickness, 
and water path. These subcolumn values are obtained from the corresponding matrix results.
For the thermodynamic phase the selected matrix element is rounded 
providing either 1 (liquid) or 0 (ice).
The particle size is derived from the cloud effective radius matrix.
The cloud optical thickness and cloud water path of each subcolumn 
is derived by integrating the sum of positive matrix elements between
the model level of the uppermost cloud layer detected and the lowermost level.


% % \begin{minipage}[ÄUSSERE POSITION][HÖHE][INNERE POSITION]{BREITE}
% % 	Beispieltext
% % \end{minipage}
% CER, CWP, COT only daytime
\begin{figure}[!ht]
\centering
 \begin{minipage}{0.45\textwidth}
   \includegraphics[scale=0.22]{./figures/{\szamidnight}.png}
   \includegraphics[scale=0.22]{./figures/{\szamidday}.png}
 \end{minipage}
 \begin{minipage}{0.45\textwidth}
   \includegraphics[scale=0.22]{./figures/{\szamorning}.png}
   \includegraphics[scale=0.22]{./figures/{\szaevening}.png}
 \end{minipage}
 \caption[Solar zenith angle for 00, 06, 12, 18 UTC.]
 {Global distribution of the solar zenith angle (SZA) 
  for UTC 00, 06, 12, and 18 on 1$^{st}$ of \MonthYear.
  In the cloud simulator SZA is used for the determination of day- and nighttime, 
  which in turn is required to mimic satellite-based retrieval characteristics, 
  i.e. cloud optical thickness is derived from visible wavelengths. 
  Consequently, cloud effective radius and cloud water path
  are also only available for daytime conditions (i.e. SZA less than 90).} 
 \label{fig:sza}
\end{figure}


CC4CL simultaneously retrieves the coud optical thickness and cloud effective radius 
by applying the Nakajima and King approach \cite{Nakajima1990}.
This method utilizes the fact that the cloud optical thickness is strongly related 
to the reflectance of clouds at a non-absorbing wavelength 
in the visible region (0.6 or 0.8 micron), while the particle size is determined 
mostly from light absorption in a near infrared channel (1.6 or 3.7 micron).
% Furthermore ice clouds absorb more and reflect less radiation at near infrared wavelengths.
As already mentioned in Section~\ref{sec:preproc}, 
CC4CL computes the cloud water path using Equation~\ref{eq:han},
where the cloud optical thickness and cloud effective radius are input parameters.
Consequently, CC4CL retrieves cloud optical thickness, cloud effective radius, and 
cloud water path based on observations made during daytime.
% only for daytime conditions. 
The purpose of the cloud simulator is to imitate the satellite observations and hence, 
these three cloud parameters are not simulated during nighttime, 
which corresponds to grid boxes having a solar zenith angle larger than 90 degree.
Figure~\ref{fig:sza} gives an example of the solar zenith angle distribution 
for the ERA-Interim time slots at UTC 00, 06, 12, and 18 on {\gcdate}.


% ==================================================================================
%3. collect statistics, i.e. means over subcolumns and 1D, 2D histograms
\subsection{Compute summary statistics}
% ==================================================================================

% grid box averages (temp) => monthly grid box averages (final)
Averaging over the subcolumns leads to grid box means 
(see top of each panel in Fig.~\ref{fig:retrieved}) at a given diagnostic time step.
Figure~\ref{fig:clouds_12utc} represents grid mean values based on the 
reanalysis file on {\gcdate} at 12 UTC.
As already explained in the previous section, the cloud optical thickness, 
cloud water path, cloud effective radius, and the daytime fraction of liquid water cloud 
are only available for daytime observations (upper four maps). 
All other pseudo-satellite cloud parameters are also retrieved at nighttime 
(lower four maps).
% Aggregation over time
By considering all reanalysis files for a given month allows then to calculate
monthly grid cell averages.
One-dimensional (1D) and two-dimensional (2D) histograms are generated using
the individual subcolumn results rather than the temporary grid box averages 
because the downscaled results mimic the spatial resolution of a satellite pixel, 
which are used for statistical aggregation of monthly cloud products.\\
1D histograms are created for the following cloud parameters: 
CTP [hPa], CTT [K], CWP [g/m$^{2}$], COT, and CER [$\mu$m]. 
Each histogram covers the solution space of its variable with the cloud phase as 
additional dimension. The used bins are:
\begin{itemize}\setlength\itemsep{0.2em}
 \item CTP: $\{$ 1, 90, 180, 245, 310, 375, 440, 500, 560, 620, 680, 740, 800, 875, 950, 1100 $\}$
 \item CTT: $\{$ 200, 210, 220, 230, 235, 240, 245, 250, 255, 260, 265, 270, 280, 290, 300, 310, 350 $\}$
 \item CWP: $\{$ 0, 5, 10, 20, 35, 50, 75, 100, 150, 200, 300, 500, 1000, 2000, 100000 $\}$ 
 \item COT: $\{$ 0, 0.3, 0.6, 1.3, 2.2, 3.6, 5.8, 9.4, 15, 23, 41, 60, 80, 99.99, 1000 $\}$
 \item CER: $\{$ 0, 3, 6, 9, 12, 15, 20, 25, 30, 40, 60, 80 $\}$ 
\end{itemize}

\noindent
The 2D joint cloud property histograms are produced for each grid box following 
the ISCCP classification, which correlates the retrieved height and optical thickness of the cloud. 
Figure~\ref{fig:isccp} illustrates the CTP-COT partitioning that was proposed 
by Rossow and Schiffer \cite{Rossow1999} within the framework of the 
International Satellite Cloud Climatology Project (ISCCP).
Based on this scheme nine different cloud types can be categorized and hence,
analysed and inter-compared with other datasets regarding their occurrence:
\begin{itemize}\setlength\itemsep{0.2em}
 \item High clouds: Cirrus (Ci), Cirrostratus (Cs), Deep Convection (Cb)
 \item Middle clouds: Altocumulus (Ac), Altostratus (As), Nimbostratus (Ns)
 \item Low clouds: Cumulus (Cu), Stratocumulus (Sc), Stratus (St).
\end{itemize}
The widths of the CTP [hPa] and COT bins used are as follows:
\begin{itemize}\setlength\itemsep{0.2em}
 \item COT: $\{$ 0, 0.3, 0.6, 1.3, 2.2, 3.6, 5.8, 9.4, 15, 23, 41, 60, 80, 100 $\}$
 \item CTP: $\{$ 1, 90, 180, 245, 310, 375, 440, 500, 560, 620, 680, 740, 800, 875, 950, 1100 $\}$
\end{itemize}


Finally, the diagnostic simulator outputs can be compared to similar statistics from satellite 
observations, which will be shown in Section~\ref{sec:sim_vs_obs}.


\begin{figure}[!ht]
  \begin{minipage}{0.4\textwidth}
    \includegraphics[width=5cm,height=6cm]{./figures/isccp_uebersicht.png} 
  \end{minipage}
  \begin{minipage}{0.6\textwidth}
    \caption[ISCCP COT-CTP classification.]{\isccp}\label{fig:isccp}
  \end{minipage}
\end{figure}

% Pseudo-retrieved grid box averages for 1 ERA-I file, 20080701, 12 UTC
\begin{figure}[!ht]
 \begin{minipage}[c]{0.5\textwidth}
   {\includegraphics[scale=0.17]{./figures/{\cotmidday}.png}\label{fig:cot_12utc}}
   {\includegraphics[scale=0.17]{./figures/{\cwpmidday}.png}\label{fig:cwp_12utc}}
   {\includegraphics[scale=0.17]{./figures/{\cfcnormal}.png}\label{fig:cfc_12utc}}
   {\includegraphics[scale=0.17]{./figures/{\cthnormal}.png}\label{fig:cth_12utc}}
 \end{minipage}\hfill
 \begin{minipage}[c]{0.5\textwidth}
   {\includegraphics[scale=0.17]{./figures/{\cermidday}.png}\label{fig:cer_12utc}}
   {\includegraphics[scale=0.17]{./figures/{\cphmidday}.png}\label{fig:cph_12utc}}
   {\includegraphics[scale=0.17]{./figures/{\cphnormal}.png}\label{fig:cphnormal_12utc}}
   {\includegraphics[scale=0.17]{./figures/{\cttnormal}.png}\label{fig:ctt_12utc}}
 \end{minipage}
  \caption[Simulated grid box averages for {\gcdate} and 12 UTC.]
{Grid box mean cloud parameters based on one ERA-Interim reanalysis file ({\gcdate} at 12 UTC).
Left column shows top down: cloud optical thickness, cloud water path, cloud fraction and cloud top height.
Right column shows top down: cloud effective radius, daytime fraction of liquid water cloud,
fraction of liquid water clouds, and cloud top temperature.
The upper four maps represent only daytime pseudo-retrieved products, while the lower
maps show the results for day- and nighttime.}
\label{fig:clouds_12utc}
\end{figure}




