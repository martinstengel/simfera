
%\section{Description of the cloud simulator}\label{simpsimu}

This section describes the main steps of retrieving monthly mean 
Cloud\_cci-like (i.e. pseudo-satellite) observations derived from 
global atmospheric reanalysis produced by ECMWF.
The clouds simulator reads the 6-hourly (00, 06, 12, 18 UTC) 
ERA-Interim files per month containing gridbox mean vertical profiles of
\begin{itemize}
    \setlength\itemsep{0.2em}
    \item pressure level $P_{lev}$ [Pa],
    \item liquid water content $LWC$ [kg/kg]$\footnote{mass of condensate per mass of moist air}$,
    \item ice water content $IWC$ [kg/kg],
    \item cloud cover $CC$ (0-1),
    \item geopotential height $h_{geo}$ [m$^{2}/s^{2}$], and
    \item temperature $T$ [K].
\end{itemize}
A detailed description of the ERA-Interim data is given in section~\ref{sec:era}.
The following steps address the things needed to ``simulate'' the observational process.
In other words, what would a satellite see if the atmosphere had the clouds
of a climate model?


\subsection{Calculation of cloud effective radius, optical thickness and water path}

In the first step the gridbox mean liquid and ice water contents at each model level
are weighted by the cloud cover profile providing the so-called 
``in-cloud'' liquid and ice water contents. 
Thus, only the cloudy part of the grid cell is taken into account.
Based on these cloud-cover-weighted profiles the liquid ($LWP$) and ice ($IWP$) water paths
per model layer can be derived, which present the total amount of liquid and ice
water between two consecutive pressure levels in the atmosphere, respectively.
The total amount of $LWP$ and $IWP$ is referred to as cloud water path ($CWP$).

The cloud optical thickness ($\tau$) per layer is obtained by the method of \citet{Han1994}
\begin{equation}\label{eq:han}
    \tau = \frac{3}{4} \frac{CWP \cdot  Q_{ext}}{r_{e} \cdot \rho}
\end{equation}
where $CWP$ represents either $LWP$ or $IWP$ depending on the thermodynamic phase.
$Q_{ext}$ denotes the extinction coefficient, which is assumed to be 2 for water and 2.1 for ice.
The density $\rho$ is set to 1 $g/m^{3}$ for water and 0.9167 $g/m^{3}$ for ice.
For the computation of effective radii ($r_{e}$) per layer the 
ERA-Interim parameterizations are implemented in the cloud simulator.\\
The cloud droplet effective radius ($r_{e}^{liq}$) follows the method of \citet{Martin1994} and 
is defined as function of liquid water content and cloud droplet number 
concentration, which in turn depends on the wind speed.
For practical reasons the simulator uses constant values for the number of 
cloud condensation nuclei over land (300) and over sea (100).
Hence, a land-sea mask (Fig.~\ref{fig:lsm}) is required, 
which is obtained from ERA-Interim Sea Surface Temperature (Fig.~\ref{fig:sst}) 
data aggregated on a 0.5$^{\circ}$ latitude-longitude grid.
The ice crystal effective radius ($r_{e}^{ice}$) is parameterized as a function of 
temperature and ice water content based on \citet{Sun1999}, which has been
revised by \citet{Sun2001}.\\
In CC4CL the simultaneously retrieved $\tau^{liq}$ ($\tau^{ice}$) and 
$r_{e}^{liq}$ ($r_{e}^{ice}$) are combined utilizing the relationship given in 
Eq.~\ref{eq:han} for deriving $LWP$ ($IWP$) as post-processed product.
To summarize, the cloud simulator adopts the retrieval characteristic of CC4CL by
using this equation, however, the other way round for computing the 
cloud optical thickness per model layer since ERA-Interim provides the information
on the cloud water content.

%\captionsetup[subfloat]{position=top}
\begin{figure}[!ht]
  \begin{minipage}[t]{0.5\textwidth}
    \subfloat[ERA-Interim land-sea mask (LSM).]
    {\includegraphics[scale=0.26]{./figures/{lsm_era_interim_0.5_0.5}.png}\label{fig:lsm}}
  \end{minipage}
  \begin{minipage}[t]{0.5\textwidth}
    \subfloat[ERA-Interim sea surface temperature (SST).]
    {\includegraphics[scale=0.26]{./figures/{sst_era_interim_0.5_0.5}.png}\label{fig:sst}}
  \end{minipage}
  \caption[ERA-Interim land-sea mask and sea surface temperature (31/08/2015).]
{ERA-Interim auxiliary data provided on a 0.5$^{\circ}$ latitude-longitude grid.}\label{fig:lsm_sst}
\end{figure}



\subsection{Pseudo retrieval}
%Addressing the mismatch between gridbox mean and satellite footprint; %``down-scaler'' 
In the previous section the pre-processing of pseudo-satellite 
$\tau$, $r_{e}$, and $CWP$ vertical profiles has been explained,
which complement along with the original ERA-Interim 3D fields the required
input data for the pseudo retrieval.
This is the essential part of the cloud simulator responsible for producing 
output comparable to Cloud\_cci data in three main steps.

%1. down-scaling, scops (subgrid cloud overlap profile sampler)
First, the mismatch in scale between that of the ERA-Interim model grid box
($\sim$ 500 km) and the satellite footprint ($\sim$ 5 km, e.g. AVHRR) has to be addressed.
Therefore, the grid box mean profiles (mentioned above) for each latitude-longitude cell
are broken into 20 subcolumns$\footnote{
Currently each grid cell is broken into 20 subcolumns.
Later the number of subcolumns will be a function of latitude.}$ 
representing the spatial resolution of the sensor.
Figure~\ref{fig:scops} shows examples for down-scaled cloud paramater profiles 
for two different grid boxes.


% cloud profiles 
\begin{figure}[!ht]
  \centering
    \includegraphics[scale=0.17]{./figures/\snapscops_CFC1_profile.png}
    \includegraphics[scale=0.17]{./figures/\snapscops_COT1_profile.png}
    \includegraphics[scale=0.17]{./figures/\snapscops_CWP1_profile.png}\\
    \includegraphics[scale=0.17]{./figures/\snapscops_CFC9_profile.png}
    \includegraphics[scale=0.17]{./figures/\snapscops_COT9_profile.png}
    \includegraphics[scale=0.17]{./figures/\snapscops_CWP9_profile.png}
  \caption[Cloud parameter profiles.]{Cloud fraction,
cloud optical thickness and cloud water path profiles for
two different grid boxes at UTC 00 on 1$^{st}$ of \MonthYear.}
  \label{fig:profiles}
\end{figure}

Cite here COSP simulator and SCOPS, we use almost the same approach.
%These subcolumns can be thought of as discrete samples, which 
%\citet{Bodas2011}.
Describe how the model information is broken into subcolumns.
Scale COT greater than 100.
Only solar COT, CWP, CER due to CC4CL.


% subcolumn cloud distribution 
\begin{figure}[!ht]
  \centering
  \begin{minipage}[c]{0.4\textwidth}
    \includegraphics[scale=0.21]{./figures/\snapscops_CFC1.png}
    \includegraphics[scale=0.21]{./figures/\snapscops_CPH1.png}
    \includegraphics[scale=0.21]{./figures/\snapscops_CER1.png}
    \includegraphics[scale=0.21]{./figures/\snapscops_COT1.png}
    \includegraphics[scale=0.21]{./figures/\snapscops_CWP1.png}
  \end{minipage}\hspace*{1cm}
  \begin{minipage}[c]{0.4\textwidth}
    \includegraphics[scale=0.21]{./figures/\snapscops_CFC9.png}
    \includegraphics[scale=0.21]{./figures/\snapscops_CPH9.png}
    \includegraphics[scale=0.21]{./figures/\snapscops_CER9.png}
    \includegraphics[scale=0.21]{./figures/\snapscops_COT9.png}
    \includegraphics[scale=0.21]{./figures/\snapscops_CWP9.png}
  \end{minipage}
  \caption[Down-scaled cloud parameter profiles.]{Subcolumn distribution 
of cloud fraction, phase, effective radius, optical thickness and 
water path derived from the corresponding grid box mean profiles. 
The left and right columns demonstrates two different grid cells 
at UTC 00 on 1$^{st}$ of \MonthYear.}
  \label{fig:scops}
\end{figure}




%2. search for upper-most cloud and collect cloud parameters

% cloud top parameters 
\begin{figure}[!htp]
  \begin{minipage}{\textwidth}
    \subfloat[Retrieved cloud top parameters at 29.5 latitude and 162.5 longitude.]
    {\includegraphics[width=\textwidth]{./figures/\snapscops_retrieved1.png}}
  \end{minipage}\vspace*{1cm}
  \begin{minipage}{\textwidth}
    \subfloat[Retrieved cloud top parameters at 31.5 latitude and 164.0 longitude.]
    {\includegraphics[width=\textwidth]{./figures/\snapscops_retrieved9.png}}
  \end{minipage}
  \caption[Retrieved cloud top parameters based on subcolumn distribution.]
{Retrieved cloud top parameters for two different grid boxes during daytime
at UTC 00 on 1$^{st}$ of \MonthYear.}
\label{fig:retrieved}
\end{figure}



%3. collect statistics, i.e. means over subcolumns and 1D, 2D histograms


%% CER, CWP, COT only daytime
%\begin{figure}[!ht]
%  \begin{minipage}[c]{0.5\textwidth}
%    \includegraphics[scale=0.27]{./figures/{\szamidnight}.png}
%    \includegraphics[scale=0.27]{./figures/{\szamidday}.png}
%  \end{minipage}\hfill
%  \begin{minipage}[c]{0.5\textwidth}
%    \includegraphics[scale=0.27]{./figures/{\szamorning}.png}
%    \includegraphics[scale=0.27]{./figures/{\szaevening}.png}
%  \end{minipage}
%  \caption[Solar zenith angle for 00, 06, 12, 18 UTC.]
%  {Solar zenith angle maps for 00, 06, 12, and 18 UTC on 1$^{st}$ of \MonthYear.} 
%  \label{fig:sza}
%\end{figure}
%
%


